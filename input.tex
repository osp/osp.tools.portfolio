% END OF osp.work.panik.git

\subject{Radio Panik}

OSP has designed the 2012 program for Brussels based alternative radio
station, Radio Panik. The program (\quote{the grille}) is constructed
through what is considered a type crime--- horizontally squashing the
text to make it fit. We debuted our new typeface Reglo for this job.

Next to the design of the program, we constructed a web interface that
provides a means for the employees of Radio Panik to generate cd covers
and posters in the same style.

The web interface allows one to browse and choose from the various image
categories of Wikimedia Commons, input a text and choose a paper size,
after which it will generate a design in a vector file format. The
output is an SVG file which can be further edited in proprietary and
libre image editors.

{\externalfigure[grille.jpg]} The program for 2012

{\externalfigure[grille_with_cover.jpg]} The program with front cover

{\externalfigure[interface_choose_options.png]} Web interface to
generate designs



% END OF osp.work.panik




\subject{La Balsamine}

\subsubject{Visual identity for the Balsamine theatre.}

Changer de direction, c'est souvent changer sa ligne graphique. Changer
de peau en soi, faire remarquer sa nouvelle présence, forger son image,
se faire une réputation. Mais est-ce aussi simple que cela, est-ce que
cela tient uniquement à cette façade de sa propre reconnaissance ? Ce
qui est certain, c'est que, comme chaque relation que nous voulions
entamer, l'idée du dialogue et du processus devait exister. Voilà
pourquoi, il était important de rencontrer des personnalités
philosophiquement engagées dans ces deux propositions. Instinctivement,
nous sommes allés vers Speculoos. Ici nous interrompons le texte,
extrait du blog http://osp.constantvzw.org/ : « En février Monica et
Fabien du Théâtre de la Balsamine sont arrivés chez Speculoos, et deux
heures après ils sont repartis, sans même vraiment le savoir, avec en
plus OSP et . Pierre a senti d'une part qu'une hétérogéneïté de
pratiques graphiques serait mieux à même de rendre compte de la
tentative d'involution du projet théâtral naissant de la Balsamine, et
d'autre part que, pour une fois, OSP pouvait risquer une réponse à une
demande sans qu'une complicité préalable n'existe avec le mode
expérimental qui lui est inhérent.

{\externalfigure[balsa_affiche-genese.jpg]} Poster of the Genesis
festival

{\externalfigure[balsa_affiches-saison1.png]} poster 1, season
2011--2012

{\externalfigure[balsa_affiches-saison2.png]} poster 2, season
2011--2012

{\externalfigure[balsa_affiches-saison3.png]} poster 3, season
2011--2012

{\externalfigure[Marc-de-Meyer_peintre.jpg]} Marc de Meyer painting the
programme 2011--2012, scale 1/1

{\externalfigure[Balsamine-wiki.png]} Wiki


% END OF balsamine






% END OF osp.work.acsr.git


\subject{acsr}

Atelier de Création Sonore Radiophonique sound repository.

\useURL[1][http://www.acsr.be][][http://www.acsr.be]\from[1] Visual
identity and WordPress design Stéphanie Vilayphiou, Ludi and Jérôme
Degive from Pica Pica, 2011

{\externalfigure[acsr-preview01.png]} website - post

{\externalfigure[acsr-preview02.png]} website - gallery

{\externalfigure[poster-fiction.jpg]} poster Fiction Performances

!{[}Semaine du son, janvier 2012 - flyer
recto(semaine-son\letterunderscore{}2.jpg) Semaine du son, janvier 2012
- flyer recto

{\externalfigure[semaine-son_1.jpg]} Semaine du son, janvier 2012 -
flyer verso


% END OF osp.work.acsr


\subject{Constant Variable}

\subsubject{Visual Identity for Free Libre Open Source Software Arts
Lab}

The 19th century town house at Rue Gallait 80 is now known as Constant
Variable. For three years, it will house an arts lab for free, libre and
open source software. Next to the Open Source Publishing studio, there
is an open video workshop and an open hardware workshop. The top floor
houses a residency.

OSP has created a modular visual identity for Constant Variable. Using
stickers with the VARIABLE logo and seperate keyword stickers
representing Free, Libre, Open Source, Software and arts.

The launch of Variable has coincided with the release of the Crickx
font, used in the identity---based upon the letters from a Schaerbeek
sign maker's shop.

The signage has been created using actual Crickx stock from Publi Fluor,
the stock upon which the digital font has been based.

{\externalfigure[variable_invitation_cover.png]}
{\externalfigure[variable_sticker.jpg]}
{\externalfigure[variable_door.jpg]} {\externalfigure[site.png]}


% END OF osp.work.gallait



% END OF osp.work.pzi.catalogue2011.git


\subject{PZI CATALOGUE 2011}

Design of the graduation flyer and catalogue of MA Networked Media of
the Piet Zwart Institute, Rotterdam.

Bound with removeable black brads, the catalogue is composed of one
stapled introduction booklet, followed by 6 leaflet/posters, one for
each student. The purple poster ink seems to be absorbed by the red
leaflet side. Each booklet and leaflet can be distributed individually.

We used public domain patent drawings to illustrate each project with a
shift of humour.

\thinrule

Software: GNU/Linux, Scribus, Gimp, Inkscape, ImageMagick, Fontforge,
PodofoImpose\ldots{} Fonts: Univers Else and Crickx. Printed at
Cassochrome (BE). {\externalfigure[recto1.png]} Flyer half-recto.

{\externalfigure[verso.png]} Flyer verso.

{\externalfigure[cat_1.jpg]} Catalogue cover. Picture by Silvio Lorusso.

{\externalfigure[cat_31.jpg]} Catalogue spread. Purple shapes are the
\quotation{overage} ink from the back side which is the poster side.
Picture by Silvio Lorusso.

{\externalfigure[cat_4.jpg]} Close-up on a patent drawing. Picture by
Silvio Lorusso.


% END OF osp.work.pzi.catalogue2011




\subject{LABtoLAB}

\subsubject{Media lab}

http://www.labtolab.org logo design and wiki theme OSP - Millle, 2011

{\externalfigure[labtolab_logo.png]} logo

{\externalfigure[Screenshot-7.png]} wiki preview 01

{\externalfigure[Screenshot-8.png]} wiki preview 02


% END OF osp.millle.LABtoLAB





\subject{Tot Later}

\subsubject{Novel}

The book was designed and typeset by OSP and printed by Holland
Ridderkerk. - Antoine Begon, Ludi, Pierre Huyghebaert, Pierre Marchand
2011 http://www.adashboard.org/?p=804

{\externalfigure[tot-later2.jpg]} overview of the interior text wrap 01

{\externalfigure[tot-later4.jpg]} book cover

{\externalfigure[tot-later5.jpg]} overview of the interior text wrap 02


% END OF osp.work.totlater






\subject{Crickx}

OSP-Crickx is a digital reinterpretation of a set of adhesive letters.

\placefigure[here,nonumber]{}{\externalfigure[crickx_header-600x251.png]}

\placefigure[here,nonumber]{}{\externalfigure[vitrine-600x407.jpg]}

\placefigure[here,nonumber]{}{\externalfigure[tailles-600x382.jpg]}

\placefigure[here,nonumber]{}{\externalfigure[tiroir-600x410.jpg]}


% END OF httpospublishconstantvzworgfoundryp322-crickx

\subject{OSP-DIN}

\useURL[1][http://ospublish.constantvzw.org/foundry/wp-content/uploads/OSP-%20DINscreenshot.png][][]\from[1]

First cut of the open source DIN, from drawing of 1932.

\useURL[2][http://ospublish.constantvzw.org/news/osp-full-scale-in-beaubourg][][http://ospublish.constantvzw.org/news/osp-full-scale-in-
beaubourg]\from[2]

\placefigure[here,nonumber]{}{\externalfigure[OSP-DINscreenshot-600x322.png]}

\placefigure[here,nonumber]{407}{\externalfigure[407.JPG]}


% END OF httpospublishconstantvzworgfoundryp32-osp-din





