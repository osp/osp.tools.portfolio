\subject{CONTEXT}

\subsubject{STRUCTURE DU DOSSIER DE TRAVAIL}

\startitemize
\item
  styles-balsa.tex -\lettermore{} styles
\item
  programme2011--12.tex -\lettermore{} le programme 2011--2012,
  englobant tous les articles (composants)
\item
  c00-edito.tex -\lettermore{} composant n°00 edito
\item
  c01-martine.tex -\lettermore{} composant n°01 martine
\item
  \ldots{}
\item
  festival\letterunderscore{}genese.tex -\lettermore{} à venir: le
  programme du festival Genèse
\item
  c00-\ldots{}
\item
  c01-\ldots{}
\item
  \ldots{}
\stopitemize

\subject{EXPORTER LE GIT LOG}

git log ---show-notes ---date=local ---reverse
---pretty=format:\quotation{\%ad\%n\%n\%an\%n\%n\%s \%n} \lettermore{}
gitlog.txt

\subject{TRANSFORMER PSD EN TRACÉS}

\startitemize
\item
  je me suis positionné sur chaque calque un par un
\item
  puis \lettermore{} menu layer \lettermore{} alpha to selection et mask
  to selection
\item
  ensuite depuis l'onglet paths dialog
\item
  \startblockquote
  j'appuie sur le bouton \quotation{selection to path}
  \stopblockquote
\item
  et là dans le menu des paths
\item
  je demande export path
\item
  et ce qui est bien c'est que tu peux exporter tous les path d'un coup
\item
  direct svg
\stopitemize

\subsubject{EN UTILISANT LE PLUGIN GIMP}

\startitemize
\item
  copier/coller le script layers2svg.py dans
  \lettertilde{}/.gimpX.X/plug-ins/
\item
  lancer gimp
\item
  aller dans Filtres/Custom/LAYERS2SVG
\item
  les 2 premiers champs ne servent à rien
\item
  copier/coller le chemin de l'image originale
\item
  écrire le chemin du fichier SVG final
\item
  lancer le script!
\item
  ouvrir le fichier svg avec gedit (ou autre)
\item
  rechercher tout le brol entre 2 paths (détecter la fin d'un svg et le
  début d'un autre) et remplacer par du vide: \quotation{}"

   \quotation{}"
\item
  sauvegarder et ouvrir le svg avec inkscape!:
\stopitemize


% END OF balsamine


A word only lives once. Each word of a text is erased except its first
occurrence. The text becomes more and more empty. Only the
\quotation{original} words are visible. The script here is applied to
.srt subtitle file, but it can easily be adapted for any text.

\subject{About the project}

This piece has been written during one of the
\useURL[1][http://www.activearchives.org/][][Active Archives]\from[1]
workshop.

\subject{Usage}

\starttyping
./disappearance.py in.srt > out.srt
\stoptyping

where

in.srt : is the srt file to process, and out.srt : is the srt file to
save the result in.

You can then use the created subtitle file with your regular movie
player. For instance, using mplayer:

\starttyping
mplayer movie.ogv -sub out.srt -subalign 1 -subpos 50 -subfont-text-scale \ 
7 -utf8 -ass -font "NotCourierSans"
\stoptyping



% END OF disappearance


