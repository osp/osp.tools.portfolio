\subject{Visual identity for the Balsamine theatre.}

\subsubject{website, programmes, posters, fanzine and more}

Changer de direction, c'est souvent changer sa ligne graphique. Changer
de peau en soi, faire remarquer sa nouvelle présence, forger son image,
se faire une réputation. Mais est-ce aussi simple que cela, est-ce que
cela tient uniquement à cette façade de sa propre reconnaissance ? Ce
qui est certain, c'est que, comme chaque relation que nous voulions
entamer, l'idée du dialogue et du processus devait exister. Voilà
pourquoi, il était important de rencontrer des personnalités
philosophiquement engagées dans ces deux propositions. Instinctivement,
nous sommes allés vers Speculoos. Ici nous interrompons le texte,
extrait du blog http://osp.constantvzw.org/ : « En février Monica et
Fabien du Théâtre de la Balsamine sont arrivés chez Speculoos, et deux
heures après ils sont repartis, sans même vraiment le savoir, avec en
plus OSP et . Pierre a senti d'une part qu'une hétérogéneïté de
pratiques graphiques serait mieux à même de rendre compte de la
tentative d'involution du projet théâtral naissant de la Balsamine, et
d'autre part que, pour une fois, OSP pouvait risquer une réponse à une
demande sans qu'une complicité préalable n'existe avec le mode
expérimental qui lui est inhérent.

{\externalfigure[balsa_affiches-saison1.png]}
{\externalfigure[balsa_affiches-saison2.png]}
{\externalfigure[balsa_affiches-saison3.png]}
{\externalfigure[Marc-de-Meyer_peintre.jpg]}


% END OF balsamine


\subject{LABtoLAB}

\subsubject{Media lab}

http://www.labtolab.org logo design and wiki theme OSP - Millle, 2011

\placefigure[here,nonumber]{logo}{\externalfigure[labtolab_logo.png]}

Logo

\placefigure[here,nonumber]{wiki preview
01}{\externalfigure[Screenshot-7.png]}

wiki preview 01

\placefigure[here,nonumber]{wiki preview
02}{\externalfigure[Screenshot-8.png]}

wiki preview 02


% END OF osp.millle.LABtoLAB


\subject{Constant Variable}

\subsubject{Visual Identity for Free Libre Open Source Software Arts
Lab}

The 19th century town house at Rue Gallait 80 is now known as Constant
Variable. For three years, it will house an arts lab for free, libre and
open source software. Next to the Open Source Publishing studio, there
is an open video workshop and an open hardware workshop. The top floor
houses a residency.

OSP has created a modular visual identity for Constant Variable. Using
stickers with the VARIABLE logo and seperate keyword stickers
representing Free, Libre, Open Source, Software and arts.

The launch of Variable has coincided with the release of the Crickx
font, used in the identity---based upon the letters from a Schaerbeek
sign maker's shop.

The signage has been created using actual Crickx stock from Publi Fluor,
the stock upon which the digital font has been based.

\placefigure[here,nonumber]{Invitation for the Opening
(Cover)}{\externalfigure[variable_invitation_cover.png]}

Invitation for the Opening (Cover)

\placefigure[here,nonumber]{Variable
sticker}{\externalfigure[variable_sticker.jpg]}

Variable sticker

\placefigure[here,nonumber]{Variable lettering
(door)}{\externalfigure[variable_door.jpg]}

Variable lettering (door)

\placefigure[here,nonumber]{Website}{\externalfigure[site.png]}

Announcement website


% END OF osp.work.gallait


\subject{Tot Later}

\subsubject{Novel}

The book was designed and typeset by OSP and printed by Holland
Ridderkerk. - Antoine Begon, Ludi, Pierre Huyghebaert, Pierre Marchand
2011 http://www.adashboard.org/?p=804

\placefigure[here,nonumber]{overview of the interior text wrap
01}{\externalfigure[tot-later2.jpg]}

overview of the interior text wrap 01

\placefigure[here,nonumber]{book cover}{\externalfigure[tot-later4.jpg]}

book cover

\placefigure[here,nonumber]{overview of the interior text wrap
02}{\externalfigure[tot-later5.jpg]}

overview of the interior text wrap 02


% END OF osp.work.totlater


